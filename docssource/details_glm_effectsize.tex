\PassOptionsToPackage{unicode=true}{hyperref} % options for packages loaded elsewhere
\PassOptionsToPackage{hyphens}{url}
%
\documentclass[]{article}
\usepackage{lmodern}
\usepackage{amssymb,amsmath}
\usepackage{ifxetex,ifluatex}
\usepackage{fixltx2e} % provides \textsubscript
\ifnum 0\ifxetex 1\fi\ifluatex 1\fi=0 % if pdftex
  \usepackage[T1]{fontenc}
  \usepackage[utf8]{inputenc}
  \usepackage{textcomp} % provides euro and other symbols
\else % if luatex or xelatex
  \usepackage{unicode-math}
  \defaultfontfeatures{Ligatures=TeX,Scale=MatchLowercase}
\fi
% use upquote if available, for straight quotes in verbatim environments
\IfFileExists{upquote.sty}{\usepackage{upquote}}{}
% use microtype if available
\IfFileExists{microtype.sty}{%
\usepackage[]{microtype}
\UseMicrotypeSet[protrusion]{basicmath} % disable protrusion for tt fonts
}{}
\IfFileExists{parskip.sty}{%
\usepackage{parskip}
}{% else
\setlength{\parindent}{0pt}
\setlength{\parskip}{6pt plus 2pt minus 1pt}
}
\usepackage{hyperref}
\hypersetup{
            pdftitle={Details: GLM effect size indices},
            pdfborder={0 0 0},
            breaklinks=true}
\urlstyle{same}  % don't use monospace font for urls
\usepackage[margin=1in]{geometry}
\usepackage{graphicx,grffile}
\makeatletter
\def\maxwidth{\ifdim\Gin@nat@width>\linewidth\linewidth\else\Gin@nat@width\fi}
\def\maxheight{\ifdim\Gin@nat@height>\textheight\textheight\else\Gin@nat@height\fi}
\makeatother
% Scale images if necessary, so that they will not overflow the page
% margins by default, and it is still possible to overwrite the defaults
% using explicit options in \includegraphics[width, height, ...]{}
\setkeys{Gin}{width=\maxwidth,height=\maxheight,keepaspectratio}
\setlength{\emergencystretch}{3em}  % prevent overfull lines
\providecommand{\tightlist}{%
  \setlength{\itemsep}{0pt}\setlength{\parskip}{0pt}}
\setcounter{secnumdepth}{0}
% Redefines (sub)paragraphs to behave more like sections
\ifx\paragraph\undefined\else
\let\oldparagraph\paragraph
\renewcommand{\paragraph}[1]{\oldparagraph{#1}\mbox{}}
\fi
\ifx\subparagraph\undefined\else
\let\oldsubparagraph\subparagraph
\renewcommand{\subparagraph}[1]{\oldsubparagraph{#1}\mbox{}}
\fi

% set default figure placement to htbp
\makeatletter
\def\fps@figure{htbp}
\makeatother


\title{Details: GLM effect size indices}
\author{}
\date{\vspace{-2.5em}}

\begin{document}
\maketitle

{
\setcounter{tocdepth}{2}
\tableofcontents
}
{ { keywords } jamovi, GLM, effect size indices, omega-sqaured,
eta-squared, epsilon-squared }

{ { GALMj version ≥ } 2.4.0 }

\hypertarget{introduction}{%
\section{Introduction}\label{introduction}}

Standardized Effect size indices produced by GLM module are the
following:

\begin{itemize}
\tightlist
\item
  \(\beta\) : standardized regression coefficients
\item
  \(\eta^2\): ``semi-partial'' eta-squared
\item
  \(\eta^2\)p : partial eta-squared
\item
  \(\omega^2\)p : partial omega-squared
\item
  \(\epsilon^2\)p : partial epsilon-squared
\end{itemize}

All coefficients but the betas are computed with the approapriate
function of the R package
\href{https://cran.r-project.org/web/packages/effectsize/index.html}{effectsize}

\hypertarget{beta-beta}{%
\section{\texorpdfstring{\(\beta\) :
beta}{\textbackslash{}beta : beta}}\label{beta-beta}}

For continuous variables, it simply corresponds to the B coefficient
obtained after standardizing all variables in the model. The
standardization of the continuous variables is done before any
transformation is applied, so if a complex model requires interaction or
polynomial terms, the terms are computed after standardization, and the
\(\beta\) are consistent.

For categorical variables, however, some comments are in order:
Categorical variables are not standardized in {GAMLj}, so the \(\beta\)
should be interpreted in terms of standardized differences in the
dependent variable between the levels contrasted by the corresponding
contrast. Consider the following example: Two groups (variable
\texttt{groups}) of size 20 and 10 respectively, are compared on a
variable Y. If one uses {GAMLj} default contrast coding
(\texttt{simple}), the B is the difference in groups means. The
\(\beta\) is the difference between the average z-scores of the
dependent variable between the two groups. Assume these are the results:

The beta is 0.352, so it means that if we compute the mean difference
between groups in the standardized \emph{y}, we obtain 0.352. In fact.

However, the \(\beta\) you obtain is not the correlation between
\emph{zy} and \emph{groups}. The correlation is 0.169:

Why is there this discrepancy? Because the groups are not balanced, so
when the correlation is computed, the variable \emph{groups} is
standardized, so the contrast coding values depend on the relative size
of the groups. The actual groups coding values used by the Pearson's
correlations are the following:

Thus, the correlation corresponds to running a regression with \emph{zy}
as dependent variables and a continuous variable featuring either -.695
or 1.390 as values. The \(\beta\) yielded by {GAMLj}, instead, is the
mean difference between X levels on the standardized Y. Please notice
that other software may yield different \(\beta\)'s for categorical
variables.

If the groups are balanced and homeschedastic, the \(\beta\) associated
with a \texttt{simple} contrast corresponds to the fully standardized
coefficient.

\hypertarget{eta2-semi-partial-eta-squared}{%
\section{\texorpdfstring{\(\eta^2\): ``semi-partial''
eta-squared}{\textbackslash{}eta\^{}2: ``semi-partial'' eta-squared}}\label{eta2-semi-partial-eta-squared}}

This is the proportion of total variance uniquely explained by the
associated effect. Being \(SS_{eff}\) the sum of squares of the effect
and \(SS_{res}\) the sum of squares residuals or of error, and
\(SS_{other}\) the sum of sum of squares associated with any other
effect in the model other than \(eff\), we have:

\[\eta^2={{SS_{eff} \over {SS_{eff}+SS_{other}+SS_{res}}}}\]

where \(SS_{eff}+SS_{other}+SS_{res}=SS_{total}\)

\hypertarget{eta2p-partial-eta-squared}{%
\section{\texorpdfstring{\(\eta^2\)p : partial
eta-squared}{\textbackslash{}eta\^{}2p : partial eta-squared}}\label{eta2p-partial-eta-squared}}

This is the proportion of partial variance uniquely explained by the
associated effect. That is, the variance uniquely explained by the
effect expressed as the proportion of variance not explained by the
other effects. Here the variance explained by the other effects in the
model is completely partialed out. Its formula is:

\[\eta^2p={{SS_{eff} \over {SS_{eff}+SS_{res}}}}\]

clearly, if there is only one independent variable, \(\eta^2=\eta^2p\)

\hypertarget{omega2p-partial-omega-squared}{%
\section{\texorpdfstring{\(\omega^2\)p : partial
omega-squared}{\textbackslash{}omega\^{}2p : partial omega-squared}}\label{omega2p-partial-omega-squared}}

This is the \emph{expected value in the population} of the proportion of
partial variance uniquely explained by the associated effect. In other
words,it is the unbiased version of \(\eta^2p\). There are different
formulas to visualize its computation, here is one. If \(df_{res}\) are
the degrees of freedon of the residual variance, \(df_{eff}\) are the
degrees of freedom of the effect, and N is the sample size, we have:

\[\omega^2p={{SS_{eff}-SS_{res} \cdot ({df_{eff}/df_{res}) \cdot }}\over{ SS_{eff}+SS_{res} \cdot [({N-df_{eff})/df_{res}}]
}}\]

It's clear that omega is similat to \(\eta^2p\), but applies a
correction for the degress of freedom. In fact, as N increases, the two
indices converge.

\hypertarget{epsilon2p-partial-epsilon-squared}{%
\section{\texorpdfstring{\(\epsilon^2\)p : partial
epsilon-squared}{\textbackslash{}epsilon\^{}2p : partial epsilon-squared}}\label{epsilon2p-partial-epsilon-squared}}

Partial Epsilon-squared is also a correction of \(\eta^2p\), but the
correction involves only the estimation of the sum of squares of the
effect, not the partial variance on which the effect is compared

\[\epsilon^2p={{SS_{eff}-SS_{res} \cdot ({df_{eff}/df_{res}) \cdot }}\over{ SS_{eff}+SS_{res}}}\]

\hypertarget{confidence-intervals}{%
\section{Confidence intervals}\label{confidence-intervals}}

In option tab \texttt{Options} it is possible to ask additional tables
for the effect size indices, containing the effect size indices and
their confidence intervals

Details for the confidence intervals computation can be found in
\href{https://github.com/easystats/effectsize}{Ben-Shachar, Makowski \&
Lüdecke (2020). Compute and interpret indices of effect size. CRAN}

Comments?

Got comments, issues or spotted a bug? Please open an issue on GAMLj at
github`` or send me an email

\end{document}
